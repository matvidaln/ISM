\documentclass[12pt,letterpaper]{article}
%\usepackage{epsfig,graphics}
\setlength{\oddsidemargin}{0in}
\setlength{\evensidemargin}{0in}
\setlength{\topmargin}{0in}
\setlength{\headheight}{0in}
\setlength{\footskip}{0.3in}
\setlength{\headsep}{0.1in}
\setlength{\textheight}{22cm}
\setlength{\textwidth}{16cm}
%\setlength{\parindent}{0in}
\setlength{\marginparsep}{0in}
\setlength{\marginparwidth}{0in}
\pagestyle{empty}
%\usepackage{multicol}
\def\gs{\mathrel{\raise1.16pt\hbox{$>$}\kern-7.0pt
\lower3.06pt\hbox{{$\scriptstyle \sim$}}}}
\def\ls{\mathrel{\raise1.16pt\hbox{$<$}\kern-7.0pt
\lower3.06pt\hbox{{$\scriptstyle \sim$}}}}

\usepackage[spanish]{babel}
\usepackage[latin1]{inputenc}
\usepackage{amsfonts}
\usepackage{amsmath}
\usepackage[pdftex]{graphicx}               %for importing graphics

\begin{document}



\begin{tabular*}{\textwidth} {@{\extracolsep{\fill}}lcr} \\ 
   {\bf  AS735-1}~2011A & {\bf Control 1} &  {\small   Prof:      Sim�n Casassus} \\
   {\small
   24  de Junio de 2011}  &  3~h & {\small \hspace{4cm} }\\
   \hline
\end{tabular*}
\begin{center}
\vspace{0.1cm} {\small (Desarrolle sus respuestas y {\bf cuide la
presentaci�n}. Con calculadora y con apuntes.) }
\end{center}

\begin{flushleft}
 \underline{Relaciones �tiles.}
Tasa de excitaci�n colisional: $\langle \sigma v\rangle \propto T^{-1/2}
\exp(-\chi/kT) $. Tasa de recombinaci�n: $\alpha \propto T^{-1/2}$. 
\end{flushleft}



%\begin{flushleft}
%\Large {\bf I} \underline{Equilibrio thermal}. 
%\end{flushleft}
%
\begin{enumerate}

%\item Modelos de region H\,{\sc ii}.

%\begin{enumerate}

\item Derivar las relaciones de Einstein-Milne (i.e. relaciones
  $C_\mathrm{ij}$ con $C_\mathrm{ji}$).

\item Escribir la ecuaci�n de balance detallado en el caso mas general
  posible, explicando en detalle cada t�rmino. 

\item Aplicar lo anterior al problema de obtener una columna para una
  especie de 2 niveles a partir de un flujo observado opticamente
  delgado. Explicar todas las aproximaciones, y tomar l�mites
  interesantes. 

\item Describir el espectro de HI, y sus mecanismos de excitaci�n.

\item Describir el espectro de H2, y sus mecanismos de excitaci�n.

\item Describir el espectro de CO, y sus mecanismos de excitaci�n.

\item Describir la estructura radial de una nube molecular hasta
  llegar a la PDR, y los mecanismos de disociaci�n de H2 .

\item Escribir el balance de ionizaci�n local en estado estacionario,
  dado un campo de intensidad espec�fica promedio $J_{\nu}(\vec{r})$. 

\item ?`Cuales son las fuentes que contribuyen a $J_{\nu}(\vec{r})$ en
  una regi�n H\,{\sc ii}? Justifique y explique en qu� consiste la
  aproximaci�n OTS.

\item ?`Cuales son los 2 principales mecanismos de enfriamiento en una
  regi�n H\,{\sc ii}, y por qu�? 

\item ?`Cuales son los principales mecanismos de calentamiento en una
  regi�n H\,{\sc ii}, y en una PDR? 

\end{enumerate}

\end{document}
